\documentclass[a4paper]{scrartcl}

\pagestyle{empty}

\usepackage{tabularx}
\newcolumntype{R}{>{\raggedleft\arraybackslash}p{2cm}}

\usepackage{luacode}
\begin{luacode}

  aufg = require "rechnen"

  function print_aufgaben()
    for i=1, 22 do
    tex.print("$")
    tex.print(rechnen.create(100,10))
    tex.print("$ & $")
    tex.print(rechnen.create(100,15))
    tex.print("$ & $")
    tex.print(rechnen.create(100,20))
    tex.print("$")
    tex.print("\\\\[2ex]")
    end
  end

\end{luacode}

\begin{document}

\begin{center}
  {\Large \textsf{Rechenblatt}}
\end{center}
\vspace{1cm}

\begin{tabular}{R@{\hspace{3cm}}R@{\hspace{3cm}}R}
  \directlua{print_aufgaben()}
\end{tabular}

\end{document}


%%% Local Variables:
%%% mode: latex
%%% TeX-engine: luatex
%%% TeX-parse-self: t
%%% TeX-auto-save: t
%%% TeX-master: t
%%% End:
